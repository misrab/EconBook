\documentclass[12pt]{memoir}

%\usepackage{cite}
\usepackage{url}

\bibliographystyle{plain}


\begin{document}
	\title{The Next Economic Era}
	\date{}
	\author{Misrab M. Faizullah-Khan}
	
	\maketitle
	
	\chapter*{Preface}
		
		There is something fundamentally dysfunctional about the way modern economies are organized throughout the world. Massive inequality pervades both developing and developed economies. 
		We are not talking about the kind of inequality we might implicitly condone alongside capitalist incentives: people trying to work harder and smarter than their peers in order to 
		have a reasonably larger share of overall wealth. We are talking about a handful of individuals commanding more wealth than millions or even billions of others, 
		despite only having the usual twenty-four hours in a day.\
		
		Many, most, or perhaps even the vast majority of people work at jobs they find unfulfilling, draining or even soul-wrenching. With the Industrial Revolution, machines  began to far outstrip mankind's 
		physical prowess. This was great for the most part, enabling huge boosts in production and relieving people from tiring physical tasks; it was also a very difficult transition for the pools of manual workers 
		that had to find new professions. In more recent times, the Information Revolution is beginning to creep into what was the realm of the human mind. More and more economic activities are being 
		performed by programs with bookkeeping efficiency orders of magnitude above what any person could achieve. Once again this revolution is proving to be two-sided: fantastic new products and services 
		for our enjoyment, and new interesting jobs for those writing software and automating processes, but also much pain in the transition.\
		
		The question then arises, if machines are more physically productive than us, and computers can be made more efficient in an increasing variety of "soft" tasks, where does that leave humans in the economic 
		mix? I certainly don't mean to dive into questions of philosophy and the meaning(s) of life; I am far too unqualified. But we absolutely must explore our changing economic roles as human beings. The 
		ultimate issue is that we are dealing with challenges society has never had to deal with in all of its history. Models of the past are simply not up to task, and if we don't start reevaluating the ways in which 
		we organize ourselves economically, we are all in for massive turmoil through inequality, unemployment, and threats to our very ways of life.\
		
		The good news is that we can use lessons from the past to logically adapt to new circumstances. In this book, I will argue what a neo-capitalistic system might look like: what its fundamental features may 
		be, how to be realistic in incorporating human nature into its design, and what its consequences might be. We have every reason to be optimistic about the future, and every reason to start planning for it now.\\
	
		% Need blank lines here to right align
		\hfill	Misrab
		
		\hfill	February 9th, 2014
		
		\hfill	Stanford, California
		% End of right align
		
		
	\chapter{The Problem}
		% Explain current problems to elicit an emotional response from people and get them hooked on the book
		
		\section{Inequality}
		
		\section{Work Dissatisfaction}
		
		\section{Na�ve Theories of Growth}
		

	\chapter{Capitalism}	
		% Now motivate some of the problems mentioned with an explanation of the system and why it would lead to them
		
	
	
	\chapter{Labor Productivity}
		% theoretical reason for wages
		
		% historical data on productivity ---> plateau?
		
		% argument for nonlinearity in modern economy anyway, externality model
		

	\chapter{Sources of Growth}
		% It not labor then what?
			% externalities and non-linearity
			% by industry: tech, automation, purely cannibalistic or value creation?
			
	%optional: \chapter{The Depreciation Factor} - doing the same may not be sustainable...
	
	\chapter{The Case for Humans}
		% what role can humans play economically and otherwise, what are our values to consider at least
		
	
	\chapter{A New Design}
		% conclusion
		
				
		
	% BIBLIOGRAPHY USING BIBTEX
	\bibliography{bibliography}
	
\end{document}
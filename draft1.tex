\documentclass[12pt]{memoir}

\usepackage{cite}


\begin{document}
	\title{The Next Economic Era}
	\date{}
	\author{Misrab M. Faizullah-Khan}
	
	\maketitle
	
	\chapter*{Preface}
		
		There is something fundamentally dysfunctional about the way modern economies are organized throughout the world. Massive inequality pervades both developing and developed economies. 
		We are not talking about the kind of inequality we might implicitly condone alongside capitalist incentives: people trying to work harder and smarter than their peers in order to 
		have a reasonably larger share of overall wealth. We are talking about a handful of individuals commanding more wealth than millions or even billions of others, 
		despite only having the usual twenty-four hours in a day.\
		
		Many, most, or perhaps even the vast majority of people work at jobs they find unfulfilling, draining or even soul-wrenching. With the Industrial Revolution machines  began to far outstrip mankind's 
		physical prowess. This was great for the most part, enabling huge boosts in production and relieving people from tiring physical tasks; it was also a very difficult transition for the pools of manual workers 
		that had to find new professions. In more recent times, the Information Revolution is beginning to creep into what was the realm of the human mind. More and more economic activities are being 
		performed by programs with bookkeeping efficiency orders of magnitude above what any person could achieve. Once again this revolution is proving to be two-sided: fantastic new products and services 
		for our enjoyment, and new interesting jobs for those writing software and automating processes, but also much pain in the transition.\
		
		The question then arises, if machines are more physically productive than us, and computers can be made more efficient in an increasing variety of "soft" tasks, where does that leave humans in the economic 
		mix? I certainly don't mean to dive into questions of philosophy and the meaning(s) of life; I am far too unqualified. But we absolutely must explore our changing economic roles as human beings. The 
		ultimate issue is that we are dealing with challenges society has never had to deal with in all of its history. Models of the past are simply not up to task, and if we don't start reevaluating the ways in which 
		we organize ourselves economically, we are all in for massive turmoil through inequality, unemployment, and threats to our very ways of life.\
		
		The good news is that we can use lessons from the past to logically adapt to new circumstances. In this book, I will argue what a neo-capitalistic system might look like: what its fundamental features may 
		be, how to be realistic in incorporating human nature into its design, and what its consequences might be. We have every reason to be optimistic about the future, and every reason to start planning for it now.\\
	
		% Need blank lines here to right align
		\hfill	Misrab
		
		\hfill	February 9th, 2014
		
		\hfill	Stanford, California
		% End of right align
		
		
	\chapter{The Problem}
		% The way capitalism functions overall
		% try for 5-10 pages
		\section{Capitalism}
			Before we start citing problems and making suggestions, it is important that we enumerate key characteristics of the current economic system. This is not easy, since there isn't just 
			one system out there. But that being said, barring remnants of Communism like North Korea or Cuba to varying degrees, there is in fact a prevailing overarching system to the global economy. That 
			system is, more or less, capitalism.\
			
			Let's then describe this term "capitalism" that we seem to hear so often in the media, before analyzing and criticizing it. As is the trend in this book, I will try to focus on the system's economic 
			realities rather than ideological undercurrents or hypotheticals. Economically speaking, capitalism is way of organizing output and consumption with free trade of property. There are many dictionary 
			definitions out there, so I have tried to phrase my own definition in a way that draws attention to what I see as the minimal "mathematical" attributes of capitalism. There's a lot of depth to these attributes, so as 
			with most things, there are two ways of reading it: superficially, or with understanding and appreciation of the subtleties and elegance. I for one have definitely committed 
			the crime of the former many times. To avoid that now, let's draw out the key points:
			
				\begin{enumerate}
					\item Capitalism is a \textbf{system}
					\item It involves the free trade of \textbf{property} or "capital"
					\item This free trade consists of both \textbf{output} and \textbf{consumption}
				\end{enumerate}
			
			The first point is that capitalism is a system. But who decided upon this system in the first place? Who is enforcing it now? If I'd like to live as a communist or something 
			else from today, could I? 
			
			
			- must understand capitalism to see its limitations
			- greed / wants, reliable and very motivating
			- humans have productive capacity
			- benefits of specialization and interdependence
			- twins: productivity and greed
			
			- small scale linearity but not large scale: large scale works in buckets i.e. lose this guy no worries, but cumulatively need marketing, finance, etc....
			- game theory behind this?
			
			- cool new visualization: each individual as blob of their productive share (their labor + capital output)
		
		% What characteristics a realistic incentive structure must have
		\section{Incentives}
		
		 % The problem of wealth distribution, with a number of illustrated examples
		 \section{A Rich Man's Game}
		 
		% The problem of job dissatisfaction, with...
		\section{My Job Sucks}
		
		% The Economic case against inequality and disillusionment (vs the moral case...)
		% e.g. skew to markets for the wealthy only, diminishing marginal returns...
		\section{All Work and No Play Makes Jack a Broken Economy}
		
	% How we are currently reacting to the problem
	\chapter{Boiling Water}
	
	% Theoretical characteristics of a solution
	\chapter{Building a Scaffold}
	
	% What role can humans play economically, in the limit
	\chapter{Hello, World}
	
	% The main barriers to implementation
	\chapter{The Naysayers}
	
	% A pragmatic roadmap for a solution
	\chapter{Realistic Optimism or Optimistic Realism}
	
\end{document}
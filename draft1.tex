\documentclass[12pt]{memoir}

%\usepackage{cite}
\usepackage{url}

\bibliographystyle{plain}


\begin{document}
	\title{The Next Economic Era}
	\date{}
	\author{Misrab M. Faizullah-Khan}
	
	\maketitle
	
	\chapter*{Preface}
		
		There is something fundamentally dysfunctional about the way modern economies are organized throughout the world. Massive inequality pervades both developing and developed economies. 
		We are not talking about the kind of inequality we might implicitly condone alongside capitalist incentives: people trying to work harder and smarter than their peers in order to 
		have a reasonably larger share of overall wealth. We are talking about a handful of individuals commanding more wealth than millions or even billions of others, 
		despite only having the usual twenty-four hours in a day.\
		
		Many, most, or perhaps even the vast majority of people work at jobs they find unfulfilling, draining or even soul-wrenching. With the Industrial Revolution machines  began to far outstrip mankind's 
		physical prowess. This was great for the most part, enabling huge boosts in production and relieving people from tiring physical tasks; it was also a very difficult transition for the pools of manual workers 
		that had to find new professions. In more recent times, the Information Revolution is beginning to creep into what was the realm of the human mind. More and more economic activities are being 
		performed by programs with bookkeeping efficiency orders of magnitude above what any person could achieve. Once again this revolution is proving to be two-sided: fantastic new products and services 
		for our enjoyment, and new interesting jobs for those writing software and automating processes, but also much pain in the transition.\
		
		The question then arises, if machines are more physically productive than us, and computers can be made more efficient in an increasing variety of "soft" tasks, where does that leave humans in the economic 
		mix? I certainly don't mean to dive into questions of philosophy and the meaning(s) of life; I am far too unqualified. But we absolutely must explore our changing economic roles as human beings. The 
		ultimate issue is that we are dealing with challenges society has never had to deal with in all of its history. Models of the past are simply not up to task, and if we don't start reevaluating the ways in which 
		we organize ourselves economically, we are all in for massive turmoil through inequality, unemployment, and threats to our very ways of life.\
		
		The good news is that we can use lessons from the past to logically adapt to new circumstances. In this book, I will argue what a neo-capitalistic system might look like: what its fundamental features may 
		be, how to be realistic in incorporating human nature into its design, and what its consequences might be. We have every reason to be optimistic about the future, and every reason to start planning for it now.\\
	
		% Need blank lines here to right align
		\hfill	Misrab
		
		\hfill	February 9th, 2014
		
		\hfill	Stanford, California
		% End of right align
		
		
	\chapter{The Problem}
		% The way capitalism functions overall
		% try for 5-10 pages
		\section{Capitalism}
			Before we start citing problems and making suggestions, it is important that we enumerate key characteristics of the current economic system. This is not easy, since there isn't just 
			one system out there. But that being said, barring remnants of Communism like North Korea or Cuba to varying degrees, there is in fact a prevailing overarching system to the global economy. That 
			system is, more or less, capitalism.\
			
			Let's then describe this term "capitalism" that we seem to hear so often in the media, before analyzing and criticizing it. As is the trend in this book, I will try to focus on the system's economic 
			realities rather than ideological undercurrents or hypotheticals. Economically speaking, capitalism is way of organizing output and consumption with free trade of property. There are many dictionary 
			definitions out there, so I have tried to phrase my own definition in a way that draws attention to what I see as the minimal "mathematical" attributes of capitalism. There's a lot of depth to these attributes, so as 
			with most things, there are two ways of reading it: superficially, or with understanding and appreciation of the subtleties and elegance. I for one have definitely committed 
			the crime of the former many times. To avoid that now, let's draw out the key points:
			
				\begin{enumerate}
					\item Capitalism is a \textbf{system}
					\item It involves the free trade of \textbf{property} or "capital"
					\item This free trade consists of both \textbf{output} and \textbf{consumption}
				\end{enumerate}
			
			The first point is that capitalism is a system. But who decided upon this system in the first place? Who is enforcing it now? If I'd like to live as a communist or something 
			else from today, could I? There are many different aspects to capitalism. Some of these were developed later than others, and are adopted differently around 
			the world. Regardless, I would argue that the very core of capitalism has always been twofold, and evolved rather naturally: the use of a single currency to trade, and Adam Smith's famous 
			"invisible hand of the market" \cite{Smith1776}.\
			
			I used the term "naturally" to describe the evolution of capitalism, and it's worth noting what I mean by this as it will be a recurring way of looking at things throughout the book. What I mean is 
			there exists a tendency towards it by virtue of its inherent simplicity, or because of some aspect of generally universal human nature. Let's look at the idea of currency to make this a bit more 
			concrete. It will also give some sense to our subsequent discussion of the so called invisible hand. Currency doesn't really have much purpose in a world where each individual is completely 
			self-sufficient. If I were to produce my own clothes, food, and so on, I wouldn't really need to trade things with others. In fact, assuming people general need and want the same things, this even takes 
			into account altruistic tendencies: if I wanted to give you something as a present, I could give you something I produced. Let's take this altruism into account in our definition of "self-sufficient", just to be 
			very careful in our generalizations right from the start. There may be a world out there with completely self-sufficient intelligent creatures, each living out their monumentally independent lives, but it 
			would be very different from ours. People do in fact often want to trade. One way of doing so, and as was done for quite a while before the advent of currency, would be to barter. However as the 
			number of different items grows, the number of possible two-way trades grows combinatorially. For example a cow might be worth one hundred pens, or twenty shirts, or perhaps a couple cell phones. It 
			gets harder and harder to keep track of all these "value conversions". In addition, trades become much less liquid. If you gave me a cow, but the milkman doesn't need one, I'll have a hard time getting milk. 
			And thus, currency comes to the rescue, offering infinitely more simplicity in trade, storage of value and as a unit of account. Any one good or service could be used as currency, but with different 
			practical advantages. We just so happened to begin with rocks, seashells and precious metals, and move on to completely subject fiat currency from there. And so we see that there is a very "natural" aspect 
			to the adoption of currency throughout the world.\
			
			The fact that currency is so useful leads people to want to use it. Of course, there must be an overall "trust" in money. If people knew the government would just keep printing money every time they needed it, 
			or would not sufficiently control its circulation, they may resort to other means of exchange. Few people like high inflation, which erodes the value of hard earned cash. In fact in collapsing economies, for instance 
			during a civil war, people have been known to return to bartering. So to answer our questions as far as currency is concerned, it is a system "enforced" through decentralized incentives, albeit with strong 
			backing from the government.\
			
			We are now starting to see capitalism's defining characteristics as an economic system. Now that we have established currency as a means to channel economic value, much like blood is channeled 
			through the human body, let's look at where incentives drive this blood: the so-called "invisible hand" of the market. Why do we care? Because this will ultimately allow us to determine the consequences 
			of a capitalist system today and in the future, from first principle instead of analogy or guessing. We will see that the outcomes we are used to from such a system are based on realities of the past. 
			As Friedrich von Hayek, one of the great classical liberalists of economics, once said of John Maynard Keynes, another father of economics, it is as if the prevailing theory had been raised to a level of 
			sanctity where criticism had become sacrilege \cite{HayekOnKeynes1}.\
			
			Adam Smith's invisible hand refers to the supposedly self-regulating aspect of markets. The beauty is that this regulation does not require very much central guidance at all, and so scales extremely well. We will 
			find that scalability today means a whole different thing from scalability in the past, and this too will be a central idea as we progress. The limitations of this invisible hand are in what exactly 
			is regulated, and when. Regulation refers to the fact that prices will often adjust until the market "clears", or all the supply is used up. This is a very, very deep concept, because it also means that rather than 
			asking people where resources should be placed on aggregate, it channels resources based on basic incentives. That's pretty abstract, so let's make it a bit more concrete.\
			
			Let's say I'm selling one hundred sheep. Say each sheep cost me a dollar to raise, in food for example. I spent a lot of time and effort raising these sheep, so I'd like to charge ten dollars for them. I'm pretty 
			lucky, so I end up selling fifty sheep. But after that initial sale, I've run out of people willing to pay that price. I'm willing to as low as one dollar per sheep, and even lower if I'd like to get rid of them at a loss, 
			so I'll keep reducing my price until all sheep are sold. Lo and behold, prices adjusted to clear the market! Conversely, if people wanted more sheep than I had, in theory I could keep raising prices 
			until the number of people willing to pay that price is as small as possible while still selling all my sheep. That is a bit simplified, since there is a trade off between percentage price change and 
			percentage volume change, and I'll probably tweak it to maximize profit, but the general idea is the same. Notice that self-regulation does not take into account morals or value-systems. The market could very 
			well be that of heroine for drug abusers. One might want to think twice before applauding the invisible hand in that case. Secondly, there can always be imperfections in the market in terms of information and 
			time. It may take a while for prices to adjust, there may be external factors limiting that adjustment, and  information on the adjustment may take time to disseminate. Those points are not necessarily mutually 
			exclusive, but in fact interrelated. Related to self-regulation is the idea that resources are naturally channelled based on market forces. Those are some fancy words, and they tie in very well with the second 
			point on capitalism.\
			
			The second point is that capitalism involves the free trade of property, or "capital". One the key ideas we'll see here is that property can be either for consumption, production, or both. As an individual, I 
			personally often tend to think of property as a consumer: tables, televisions, food. It's worth remembering that a factory machine is very much property as well. In fact so is a bond or share in a company, 
			which makes our conversation on the free trade of property extremely interesting!\

			Economics is often introduced as the science of scarce resources in the face of unlimited wants, leading to questions of what to produce, how and when to produce it, and whom to produce it for. Tacit in this can 
			be what people consume, particularly in a system that involves central planning. For example in the Soviet economy, it was probably difficult to start using your factory for making toys even if you 
			saw "demand" for this, if the government had assigned you to the production of bullets. The poor little demanders of toys would be left wanting. Once you have a decentralized system like capitalism, with the 
			invisible hand of the market naturally adjusting prices, simply add the free trade of property and the result is fantastic dynamism. Children would beg their parents for toys, entrepreneurial producers would 
			detect that want one way or another, and they would be free to use their productive capital such as machines to seize the opportunity. In fact, they might need to borrow money to invest in these machines. 
			They or their financial intermediary such as a bank would put up a loan, a piece of property itself, for sale. This loan might have a fixed coupon, an extra amount of money to be paid back as compensation for
			borrowing. Different investors would bid for this bond, changing the bonds price and the resultant percentage value of the fixed coupon. Depending on how much return investors are seeking, their alternatives, 
			their perceived risk, an interest rate would be determined. Or perhaps the bank would determine this from the onset, and make the loan directly. This is just a modicum of the possible complexity in a capitalist 
			system. Try and implement such a sophisticated mechanism with central planning, and you'll get all sorts of distortions: loans going to economically unreasonable places, too many shoes being made in the face 
			of famine, and so on. That's not to say capitalism is perfect - far from it. But it is crucial that we understand early on that the idea of decentralized incentives, alongside the free trade of property that enables 
			people to act economically based on these incentives, is a naturally scalable system. Any serious criticism of capitalism will recognize the strengths as well as limitations of its different characteristics. Only then will 
			we be able to try to design a system with the characteristics we like, and without the ones we have learnt to dislike. If we proceed carefully, it shouldn't be too different from cooking!\
			 
			The third point will arguably prove to be the most important to our thesis: the fact that the free trade of property involves two sides, production and consumption. It is so important that it will probably 
			take us a couple chapters to truly develop a deep understanding of the issue. Nevertheless, it is worth having a first glance. A key feature of a capitalist system is that it incentives people, to a certain 
			extent, based on greed. This notion of "greed" can be quite innocent. It involves our needs, such as food and hopefully shelter, and our wants, which can be anything from a cup of tea to less modest pursuits 
			like a private jet. The useful aspect of this desire for things others produce is that it is generally widespread, and sufficiently motivating for people to work. Effort and reward certainly do not always go 
			hand in hand linearly, and so it is not a simple relationship. But overall, in such a system if you want things others produce, you'll have to produce yourself. I argue that the subtleties of this create some of 
			the most profound problems with capitalism today. It will take me several chapters to attempt to elaborate on this, but I hope that will allow me to convey the main ideas with clarity. Before we do that, let's 
			have a slightly deeper look at our incentives to work and consume, since they are so central to a capitalist system, and will in all likelihood be so for a new economic system.
		
		% What characteristics a realistic incentive structure must have
		\section{Incentives}
			% i.e. now that we know a bit more about capitalism, reasons it used to and once again has superseded central planning
			
			
		
		 % The problem of wealth distribution, with a number of illustrated examples
		 \section{A Rich Man's Game}
		 
		% The problem of job dissatisfaction, with...
		\section{My Job Sucks}
		
		% The Economic case against inequality and disillusionment (vs the moral case...)
		% e.g. skew to markets for the wealthy only, diminishing marginal returns...
		\section{All Work and No Play Makes Jack a Broken Economy}
		
	% How we are currently reacting to the problem
	\chapter{Boiling Water}
	
	% Theoretical characteristics of a solution
	\chapter{Building a Scaffold}
	
	% What role can humans play economically, in the limit
	\chapter{Hello, World}
	
	% The main barriers to implementation
	\chapter{The Naysayers}
	
	% A pragmatic roadmap for a solution
	\chapter{Realistic Optimism or Optimistic Realism}
	
	
	% BIBLIOGRAPHY USING BIBTEX
	\bibliography{bibliography}
	
\end{document}